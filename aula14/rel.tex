% -------------------------------------------------------------
% Arquivo :  relatório modelo
% -------------------------------------------------------------
% O percentual(%) serve para incluir comentários:
% tudo o que fica à direita dele não é interpretado pelo LaTex
% Linhas e espaços em branco também **NÃO** são
% interpretadas pelo LaTex

%% As intruções seguintes são o cabeçalho e devem estar antes do
%% \begin{document}

%\documenclass: mandatorio, indica o tipo/formato de documento
\documentclass[brazilian,12pt,a4paper,final]{article}
% tamanhos de fontes: 10pt, 11pt ou 12pt
% opções de estilo (padrões): article, report, book, slide, letter (artigo, relatorio, livro, apresentação de slides, carta)




%% Pacotes extras (opcionais):

% *babel* contem as regras de ifenização
\usepackage[brazil]{babel}

% *t1enc* permite o reconhecimento dos acentos inseridos com o teclado
%\usepackage{t1enc}

% *inputenc* com opção *utf8* permite reconhecimento dos caracteres com codificação UTF8, que é padrão dos esditores de texto no Linux. Isso permite reconhecimento automático de acentuação.
\usepackage[utf8]{inputenc}


% *graphicx* é para incluir figuras em formato eps
\usepackage{graphicx} % para produzir PDF diretamente reescrever esta linha assim: \usepackage[pdftex]{graphicx}

% *color* fontes soloridas
\usepackage{color}
%%% fim do cabecalho %%%

\pagestyle{empty}
\title{Métodos Computacionais da Física A}
\author{Aluno: Leonardo Machado Barcelos - Matrícula: 00302060 \\ IF-UFRGS}

\begin{document}

%Abaixo podem ver como se deve colocar letras acentuadas ou latinas se
%o pacote *t1enc* não dfosse usado
\section{Interpolação}
% Aqui a Introdução \c{c} e \~a  é a forma standar  de escrever
% carateres ASCII extendidos (acentos, etc), porem com o pacote t1enc
% declarado acima podemos escrever diretamente ç em lugar d \c{c}, etc

\begin{figure}[hbtp]
\begin{center}
\includegraphics[width=8cm]{nev_.pdf}
\caption{Curva gerada com o método de Neville}
\label{fig}
\end{center}
\end{figure}

\begin{center}
    \begin{table}[h]
        \begin{tabular}{|c|c|} \hline
            x & y \\ \hline
            -8.0 & 3.5 \\ \hline
            -1.0 & 4.0 \\ \hline
            2.5 & 10.3 \\ \hline
            3.4 & -12.9 \\ \hline
            5.6 & 9.8 \\ \hline
            10.1 & 12.1 \\ \hline
            11.2 & 20.9 \\ \hline
            15.0 & 2.1 \\ \hline
        \end{tabular}
    \caption{Pontos Utilizados para fazer a interpolação}
    \end{table}
\end{center}


\end{document}

